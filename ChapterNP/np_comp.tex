\chapter{NP-Completeness}
\lettrine[lines=4]{N}{P-Complete} is a way of classifying the complexity of a problem in terms of how difficult the problem is, or how `long' it takes to run. For our purposes, the problems will be limited to decision problems, where the solution is given as either a `yes' or `no'. Before we can discuss NP-Complete problems, we should first go over some definitions.\\

\section{Definitions}
\begin{description}
    \item[Deterministic:] \index{Deterministic}An algorithm is \textbf{deterministic} if it follows a single path from input to outcome.
    \item[Nondeterministic:]  \index{Nondeterministic}An algorithm is \textbf{nondeterministic} if it can follow a branching path that can go from one input to multiple outcomes. All of these paths are potentially traversed simultaneously.
    \item[Polynomial Time:] \index{Polynomial Time}An algorithm is said to be a \textbf{polynomial time} algorithm if its execution time scales by a polynomial. An example of this is a linear search algorithm, which can be run in O($n$) time. \textbf{Polynomial time} algorithms could also be bounded above by a polynomial, meaning that an algorithm which runs in O($\log(n)$) time is also a polynomial time algorithm, as it is better than O($n$). Examples of non-polynomial times include O($2^n$), O($n!$), or O($n^n$).
    \item[P:] \index{P}A problem is classified as \textbf{P} if it can be solved by a deterministic turing machine in polynomial time.
    \item[NP:] \index{NP}A problem is classified as \textbf{NP} if a potential solution can be verified by a deterministic turing machine in polynomial time. As each solution is verifiable in polynomial time, if we were to build a non-deterministic turing machine we could try every possible solution at the same time, verifying each of them. This is what it means to say that \textbf{NP} problems are decidable in non-deterministic polynomial time. The set of problems \textbf{NP} contains the set of problems \textbf{P}. There is also a very famous question to whether or not the entirety of \textbf{NP} lies in \textbf{P}.
    \item[NP-Hard:] \index{NP-Hard}A problem is classified as \textbf{NP-Hard} if it can be reduced to any other \textbf{NP-Hard} problem in polynomial time.
    \begin{figure}[H]
        \centering
        \includegraphics[width=0.4\textwidth,valign=t]{ChapterNP/complexity.pdf}
        \includegraphics[width=0.4\textwidth,valign=t]{ChapterNP/complexity2.pdf}
    \end{figure}
    \item[NP-Complete:] \index{NP-Complete}A problem is classified as \textbf{NP-Complete} if it is both \textbf{NP-Hard} and \textbf{NP}.
    \item[Reduction:] A \textbf{reduction} \index{Reduction}is a method for using problems of known difficulty to show that another problem is at least as difficult. If one can perform a reduction in polynomial time from a known NP-Complete problem to some new problem, this new problem must be at least as complex as that NP-Complete problem.
    \item[Gadget:] A \textbf{gadget} \index{Gadget}is a mapping from the input of one problem to the input of another problem. \textbf{Gadgets} are used in construction reductions.
\end{description}

\section{Founding Papers}
\subsection{Cook-Levin Theorem (1971)}
    \index{Cook-Levin Theorem}The first problem that was known to be NP-Complete is called the Boolean Satisfiability problem, or SAT. The theorem that declares SAT as NP-Complete is called the Cook-Levin Theorem, named after Steven Cook and Leonid Levin. The two worked in parallel writing their own papers, with Cook in the United States and Levin in the Soviet Union. 
\subsection{Karp's 21 Problems (1972)}
    \index{Karp's 21 Problems}In 1972, Richard Karp used the Cook-Levin theorem that SAT is NP-Complete to put together a list of 21 other problems that are also NP-Complete. He showed these problems as NP-Complete by performing polynomial time reductions from SAT to each of the 21 problems. Karp's 21 problems are as follows, with nesting indicating which reductions he used.


    %\newlist{longitemize}{itemize}{9}
    \setlistdepth{9}
    \setlist[itemize,1]{label=\textbullet}
    \setlist[itemize,2]{label=\textbullet}
    \setlist[itemize,3]{label=\textbullet}
    \setlist[itemize,4]{label=\textbullet}
    \setlist[itemize,5]{label=\textbullet}
    \setlist[itemize,6]{label=\textbullet}
    \setlist[itemize,7]{label=\textbullet}
    \setlist[itemize,8]{label=\textbullet}
    \setlist[itemize,9]{label=\textbullet}
    \renewlist{itemize}{itemize}{9}

    \begin{itemize}
        \item SAT
        \begin{itemize}
            \item 0-1 Integer Programming
            \item Clique
            \begin{itemize}
                \item Set Packing
                \item Vertex Cover
                \begin{itemize}
                    \item Set Covering
                    \item Feedback Node Set
                    \item Feedback Arc Set
                    \item Directed Hamiltonian Circuit
                    \begin{itemize}
                        \item Undirected Hamiltonian Circuit  
                    \end{itemize}
                \end{itemize}
            \end{itemize}
            \item 3-SAT
            \begin{itemize}
                \item Chromatic Number  
                \begin{itemize}
                    \item Clique Cover
                    \item Exact Cover
                    \begin{itemize}
                        \item Hitting Set
                        \item Steiner Tree
                        \item 3-Dimensional Matching
                        \item Knapsack Problem
                        \begin{itemize}
                            \item Job Sequencing
                            \item Partition
                            \begin{itemize}
                                \item Max Cut
                            \end{itemize}
                        \end{itemize}
                    \end{itemize}
                \end{itemize}
            \end{itemize}
        \end{itemize}
    \end{itemize}

\section{Problems}
\subsection{Boolean Satisfiability}
    The \textbf{Boolean Satisfiability}\index{Boolean Satisfiability} decision problem (\textbf{SAT}) is to determine if there is a set of values for which a given boolean formula evaluates to true. These formulae can contain the operations \textbf{and} ($x_1 \land x_2$), \textbf{or} ($x_1 \lor x_2$), or \textbf{not} ($\overline{x_1}$). An example of one such formula can be seen below. We will take the Cook-Levin theorem as our proof that \textbf{SAT} is \textbf{NP-Complete}. 
    \begin{align*}
        (x_1 \lor x_2) \land (\overline{x_1}) \land (\overline{x_2} \lor x_3 \lor x_1)
    \end{align*}

\subsection{3-Satisfiability}
    The \textbf{3-Satisfiability} \index{3-Satisfiability}decision problem (\textbf{3-SAT}) is similar to the \textbf{SAT} problem, but with the added restriction that the boolean formula must be in the form 3-conjunctive normal form (\textbf{3-CNF}). \textbf{CNF} states that clauses are sets of \textbf{or} operations separated by \textbf{and} operations. There is no limit on the number of clauses which the \textbf{and} operations operate on. \textbf{3-CNF} requires that each clause has two \textbf{or} operations performed on three variables. An example of a formula in \textbf{3-CNF} can be seen below.
    \begin{align*}
        (x_1 \lor x_2 \lor x_3) \land (\overline{x_1} \lor x_2 \lor \overline{x_3}) \land (x_1 \lor \overline{x_2} \lor x_3)
    \end{align*}
    \subsubsection{Is it in NP?}
        To show that \textbf{3-SAT} is in \textbf{NP}, we must show that if given a potential solution to the problem we can verify the solution in polynomial time. A potential solution can be represented with a value for each of the variables in the formula. Plugging values into and verifying the formula is a polynomial time operation, so the \textbf{3-SAT} problem is in \textbf{NP}.
    \subsubsection{Is it in NP-Hard?}
        A boolean formula can be quickly converted to \textbf{3-CNF}. Clauses of fewer than three variables can be grown to contain three variables by simply adding a second copy of a variable already in the clause.
        \begin{align*}
            (\overline{x_1}) = &(\overline{x_1} \lor \overline{x_1} \lor \overline{x_1}) \\
            (x_1 \lor x_2) = &(x_1 \lor x_1 \lor x_2) = (x_1 \lor x_2 \lor x_2)
        \end{align*}
        Clauses of greater than three variables can be broken down into smaller clauses by introducing addition variables. The purpose of these new variables is to spread the one clause out without changing its meaning. An existing clause of $n$ variables can be reduced to a collection of $n-2$ clauses in \textbf{3-CNF} with $n-3$ additional variables.
        \begin{align*}
            &(x_1 \lor x_2 \lor ... \lor x_{n-1} \lor x_{n}) = \\ &(x_1 \lor x_2 \lor z_1) \land (\overline{z_1} \lor x_3 \lor z_2) \land ... \land (\overline{z_{n-4}} \lor x_{n-2} \lor z_{n-3}) \land (\overline{z_{n-3}} \lor x_{n-1} \lor x_{n})
            %(x_0 \lor x_1 \lor x_2 \lor x_3 \lor x_4) = (x_0 \lor x_1 \lor z_0) \land (\overline{z_0} \lor x_2 \lor z_1) \land (\overline{z_1} \lor x_3 \lor x_4)
        \end{align*}
        This conversion to \textbf{3-CNF} can be done in polynomial time, meaning \textbf{SAT} is reducible to \textbf{3-SAT} in polynomial time. This means that \textbf{3-SAT} is in \textbf{NP-Hard}.

\subsection{Directed Hamiltonian Cycle}
    The \textbf{Directed Hamiltonian Cycle} \index{Directed Hamiltonian Cycle}decision problem is to determine if there is a Hamiltonian cycle in a given directed graph. That is to say, does there exist a cycle that passes through every vertex of the graph exactly once. We will now prove that the \textbf{Directed Hamiltonian Cycle} problem is an \textbf{NP-Complete} problem. To do this, we will first show that it is in \textbf{NP}. Then we will show that the problem is in \textbf{NP-Hard}.
    \subsubsection{Is it in NP?}
        To show that the \textbf{Directed Hamiltonian Cycle} problem is in \textbf{NP}, we must show that if given a potential solution to the problem we can verify the solution in polynomial time. We will represent the solution with a list of vertices, representing the vertices of the cycle in the order that they occur. We first check that each vertex of the graph is in the list exactly once. Then we will ensure that each pair of vertices adjacent in the list are also adjacent in the graph. Both of these operations can be done in polynomial time. As we can verify a solution in polynomial time, the \textbf{Directed Hamiltonian Cycle} problem is in \textbf{NP}.
    \subsubsection{Is it in NP-Hard?}
        To show that the \textbf{Directed Hamiltonian Cycle} problem is in \textbf{NP-Hard}, we must show that we can reduce a known \textbf{NP-Hard} problem to the \textbf{Directed Hamiltonian Cycle} problem in polynomial time. We will be using \textbf{3-SAT} for this proof. Let the clauses be represented by $c_1, c_2, ..., c_k$ and the variables be represented by $x_1, x_2, ..., x_L$. For our gadget, we will build a directed graph based on the given clauses and variables. We will build $L$ of the below structures, with $2*k$ vertices in each of the crossbar. 
        \begin{center}
            \includegraphics[width=0.6\textwidth]{ChapterNP/gadget1.pdf}
        \end{center}
        Each of the structures will represent one variable, attaching the bottom of one to the top of another. Each clause will be given a single vertex away from the main structure.
        \begin{center}
            \includegraphics[width=0.9\textwidth]{ChapterNP/gadget2.pdf}
        \end{center}
        The final step is to connect each of the variables to a clause if and only if the variable occurs in the clause. We connect a variable to a clause by drawing edges from one of the variable's crossbar vertices to the clause vertex, then back to a vertex adjacent to the crossbar vertex. If the variable has a \textbf{not} operation performed on it in the clause, then the adjacenct vertex should be to the left of the initial. Otherwise, it should be to the right. Once all of the required edges have been connected from variables to clauses, we can perform our check. If there exists a Hamiltonian path on this graph, then the 3-SAT problem is also satisfiable. Take the 3-SAT problem below for example.
        \begin{align*}
            (x_1 \lor x_2 \lor x_3) \land (\overline{x_1} \lor x_2 \lor \overline{x_3})
        \end{align*}
        This 3-SAT problem can be turned into the below graph. The red edges show the path of the Hamiltonian cycle. As there exists a Hamiltonian cycle, the 3-SAT problem is known to be satisfiable, and we can read which values satisfy the 3-SAT problem by examining the graph. As the edges in $x_1$'s crossbar point from left to right, we will take $x_1$'s value as true. We can repeat this, finding a value of true for $x_2$ and false for $x_3$. With this set of values, we now know a set of values for which the above 3-SAT problem is satisfiable.
        \begin{center}
            \includegraphics[width=1\textwidth]{ChapterNP/gadget-example.pdf}
        \end{center}

\subsection{Clique}
    The \textbf{Clique}\index{Clique} decision problem is to determine if there exists in a graph $G$ a complete subgraph of size $k$.
    \subsubsection{Is it in NP?}
        If given a subgraph of $G$, first determine if there are $k$ vertices. If there are $k$ vertices, then determine if they all share edges. This can be done in $O(k^2)$ time, which means this problem is polynomial verifiable.
    \subsubsection{Is it in NP-Hard?}
        To show that the \textbf{Clique} problem is \textbf{NP-Hard}, we will use a gadget to show that we can reduce \textbf{3-SAT} to \textbf{Clique} in polynomial time. We will denote each clause as a cluster of 3 vertices. For every pair of vertices, create an edge if the vertices are not in the same cluster and if the vertices are not complements of each other (ie. $X$ and $\overline{X}$). For a boolean statement of $k$ clauses, if the resulting graph contains a clique of size $k$, then the original 3-SAT problem is satisfiable. As an example, take the following 3-SAT problem.
        \begin{align*}
            (x_1 \lor x_2 \lor x_3) \land (\overline{x_1} \lor x_2 \lor \overline{x_3}) \land (\overline{x_1} \lor \overline{x_2} \lor x_3)
        \end{align*}
        In the graph we will be using the \textbf{!} symbol as our \textbf{not}. As this boolean statement contains three clauses, we are looking for a 3-clique from the resulting graph. As you can see below, there exists a clique of size three in the graph, so the original 3-SAT problem is satisfiable. By examining the graph we can see one set of values for which it is satisfiable. The connected vertices are $x_1$, $x_2$,and $x_3$; this means if we set those three values to true, it would satisfy the 3-SAT problem.
        \begin{center}
            \includegraphics[width=0.6\textwidth]{ChapterNP/clique.pdf}
        \end{center}
    
\subsection{Vertex Cover}
    The \textbf{Vertex Cover} \index{Vertex Cover}decision problem is to determine if there exists in a graph $G$ a vertex cover of size $\leq k$. A vertex cover of an undirected graph is a subset of vertices $S$ such that every edge in $G$ is adjacent to at least one vertex in $S$.
    \subsubsection{Is it in NP?}
        If given a graph $G$, a max size $k$, and a potential vertex cover $S$, we can first check that $|S| \leq k$. This is likely a constant time operation. Next we can use the following algorithm: 
        \begin{itemize}[noitemsep]
            \setlength{\itemindent}{1.25in}
            \item For each edge $(u,v) \in G$
            \begin{itemize}
                \setlength{\itemindent}{1.25in}
                \item If $u \notin S$ and $v \notin S$
                \begin{itemize}
                    \setlength{\itemindent}{1.25in}
                    \item Return $false$
                \end{itemize}
            \end{itemize}
            \item If not returned $false$, return $true$
        \end{itemize}
        This algorithm runs in $O(m)$ where $m$ is the number of edges in $G$, meaning that the vertex cover is polynomial time verifiable.
    \subsubsection{Is it in NP-Hard?}
        %https://www.geeksforgeeks.org/proof-that-vertex-cover-is-np-complete/
        To show that the \textbf{Vertex Cover} problem is in \textbf{NP-Hard}, we will perform a reduction from the \textbf{Clique} problem. We can begin by looking at $G'$, the complement of $G$. It turns out, the clique decision problem of size $k$ in $G$ is the same as the vertex cover of size $n-k$ in $G'$. This can be shown by assuming there is a clique of size $k$ in $G$. We will call the set of vertices in the overall graph $V$ and the set of vertices in the clique $V'$.  \par
        We will pick any edge $(u,v)$ from the complement graph $G'$. Either $u$ or $v$ is a member of $V'$, but not both. We can prove this by contradiction. Assume that both $u$ and $v$ are members of $V'$, then there would be an edge between them in the clique. This would mean that there would not be an edge between them in $G'$. This is a contradiction. As our starting assumption was that $u$ and $v$ were both in $V'$, at least one of the vertices must be outside of $V'$. This means that at least one of $u$ or $v$ must fall inside the set of vertices $V - V'$. \par
        Now we will assume there is a vertex cover $V''$ of size $|V|-|V'|$, or $|V|-k$, in $G'$. If we pick any edge $(u,v)$ from $G'$, both $u$ and $v$ cannot be outside of $V''$. Thus, all edges $(u,v)$ such that $u$ and $v$ are both outside of $V''$ are in G. These edges create a clique of size $k$. \par
        This means that there is a clique of size $k$ in graph $G$ if and only if there is a vertex cover of size $n-k$ in $G'$. A graph can be converted to its complement in polynomial time, thus the \textbf{Clique} problem can be reduced to the \textbf{Vertex Cover} problem in polynomial time. \par
        As an example, there is a clique of size four in the below graph.
        \begin{center}
            \includegraphics[width=1\textwidth]{ChapterNP/vertex_cover_1.pdf}
        \end{center}
        As there is a clique of size 4 in the graph above, there will be a vertex cover of size $n-4$, or 2, in the complement below. As you can see, there is a vertex cover of size 2.
        \begin{center}
            \includegraphics[width=1\textwidth]{ChapterNP/vertex_cover_2.pdf}
        \end{center}
